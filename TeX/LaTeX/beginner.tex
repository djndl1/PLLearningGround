% preamble
\documentclass[a4paper,12pt]{article}
%\usepackage{xeCJK}% for CJK

% \usepackage[T1]{fontenc} % for pdflatex, xelatex and lualatex don't require this
\usepackage{fontspec}
\usepackage{hyperref}
\usepackage{babel}

\usepackage[onehalfspacing]{setspace} % change spacing

\babelprovide[import]{chinese}
\babelfont[chinese]{rm}{FandolSong-Regular.otf}

% a flexible and easy interface to page dimensions.
\usepackage[a4paper, portrait, inner=1.5cm, outer=3cm, top=2cm, bottom=3cm, bindingoffset=1cm]{geometry}


\usepackage{microtype}
% \usepackage[utf8]{inputenc} meant only for tex or pdftex

% \newcommand{\code}{\texttt} a macro definition
% \newcommand{command}[arguments][optional]{definition}
% optional if present is the first argument
% use \newcommand to create styles!
\newcommand{\code}[1]{\texttt{#1}}
\newcommand{\keyword}[2][\ttfamily]{{#1#2}}

\newcommand{\pangram}[1][\rmfamily]{{#1 The quick brown fox jumps
over the lazy dog. 1234567890 \foreignlanguage{chinese}{中文测试}}\par}

% compactenum, inparaenum and asparaenum
% compactitem, inparaitem and asparaitem
% compactdesc, inparadesc and asparadesc
\usepackage{paralist}

% customize lists
\usepackage{enumitem}
\setlist{nolistsep} % no list separator to obtain compact lists
\setitemize[1]{label=---} % em dash for all unordered lists

% set ordered lists to use circled capital letters as labels on the first level
\setenumerate[1]{label=\textcircled{\scriptsize\Alph*}, font=\sffamily}

\usepackage{layouts}

\hyphenation{ac-ro-nym} % tell the engine how to break the word acronym


% a document environment
\begin{document}

\tableofcontents %a .toc file will be generated

\title{LaTeX Beginner Note}
\author{djndl1}
\date{\today}

% make a title along with the author and the date
\maketitle

% make a section. An article does not have a chapter
\section{Formatting Fonts, Shapes and Styles}

Text can be \emph{emphasized}, \textit{italic}, \textbf{bold}, \textsl{slanted} and even \textsc{small caps}. They can also be \textbf{\textit{nested}}.
\code{\textbackslash emph} % use typewriter/monospace font for code
is a \emph{semantic markup}, unlike the other
\code{text**} \emph{shape}
markups in that it does not refer to the appearance of text.

For headings, normally we use \textsf{Sans Serif} for readability on low-resolution devices. Standard text in \LaTeX\ is in \textrm{Roman fonts}.
We can also use \emph{declarations} like \ttfamily \textbackslash sffamily, \textbackslash ttfamily, \textbackslash bfseries, \textbackslash itshape \normalfont and \ttfamily \textbackslash rmfamily \normalfont to change the underlying font context. However, It would be so wrong to deal with them without using grouping.



\keyword[\bfseries\sffamily]{Grouping} can be so useful with {\em declarations}. Here, {\ttfamily \textbackslash em} can only be valid within the {\ttfamily scope} of the group.

\noindent Text size can be set to {\tiny tiny}, {\small small}, {\normalsize normalsize}, {\large large}, {\Large Large} and {\LARGE LARGE}.
Also, there are {\footnotesize footnotesize} and {\scriptsize scriptsize}.


\subsection{Using Environments}

\begin{large}
  For every declaration there's a corresponding environment carrying the same name except the backslash.
  Environments are like declarations with a built-in scope.
\end{large}



\subsection{Using Boxes}

% paragraph box with top alignment
\parbox[t]{1cm}{I have a narrow column here!}
\quad
\parbox[b]{1cm}{I have a narrow column here!}
\quad
\begin{minipage}{3cm}
A minipage is a \code{\textbackslash parbox} for a large amount of text.

\footnote{footnote is possible inside a \code{\textbackslash minipage}.}
\end{minipage}

\subsection{Linebreaks}

\code{\textbackslash\textbackslash} or \code{\textbackslash newline} \\[1cm] breaks the current line and they also supports a few arguments.

\code{\textbackslash linebreak} breaks a line too but keep the full justication. \code{\textbackslash nolinebreak} prevents a line break at the current position. \nolinebreak[4] \linebreak[1]
This line should not be broken.

\LaTeX\ break lines at spaces between words if meaningful. \code{\~} stands for an interword space where no break is allowed. Using \code{\textbackslash\textbackslash} and its friends is susceptible to further document changes.

\subsection{Ligatures}

Ligatures are automatic if the fonts used support this feature. Use \code{\{\}} or \code{\textbackslash/} to separate them. -- and -\/-.

A single \code{-} marks mark hyphenation or compound words. A \code{--} is used to indicate a range of some values. Wider dashes are used to mark a parenthetical thought.

\subsection{Dots}

\LaTeX\ assumes a dot after a capital letter an abbreviation and one after a lowercase a normal period. Use \code{\textbackslash @} and \code{\textbackslash } to correct that.

\subsection{Accents}

Accent commands are available. Set \code{\textbackslash [utf8]\{inputenc\}} to use UTF-8. Instead of
\'{a}, input  á.

\subsection{Justification}

\begin{center}
  We want titles \\
  to be centered \\
\end{center}

\subsection{Quotation}

A short quote goes here:

\begin{quote}
A short quote.
\end{quote}

I also have a long quotation.

\begin{quotation}
  Lorem ipsum dolor sit amet, consectetur adipiscing elit, sed do eiusmod tempor incididunt ut labore et dolore magna aliqua. Ut enim ad minim veniam, quis nostrud exercitation ullamco laboris nisi ut aliquip ex ea commodo consequat. Duis aute irure dolor in reprehenderit in voluptate velit esse cillum dolore eu fugiat nulla pariatur. Excepteur sint occaecat cupidatat non proident, sunt in culpa qui officia deserunt mollit anim id est laborum.
\end{quotation}

\section{Lists}

\subsection{Unordered Lists}

I have several items:

\begin{itemize}
  \item item1
  \item item2
  \item item3
\item item4
\end{itemize}

They can also be nested:

\begin{itemize}
  \item Topitem1
        \begin{itemize}
          \item nested item
        \end{itemize}
  \item topitem2
\end{itemize}

\subsection{Ordered Lists}

\begin{enumerate}
  \item topitem1
        \begin{itemize}
          \item unordered item1
          \item unordered item2
        \end{itemize}
  \item topitem2
        \begin{enumerate}[label=\Roman*] % set list properties at local level, * is only there for distinguising
          \item ordered item1
          \item  ordered item2
        \end{enumerate}
\end{enumerate}

Somehow I need to resume the previous list:

\begin{enumerate}[resume*]
  \item topitem3
\end{enumerate}

Using a compact list:

\begin{compactenum}
  \item Compact Item1
  \item Compact Item2
  \begin{compactitem}
    \item nested compact item 1
    \item nested compact item 2
  \end{compactitem}
\end{compactenum}

\subsection{Definition/Description Lists}

There are several useful packages for lists.

\begin{description}
  \item[paralist] provides compact lists and list versions that
can be used within paragraphs, helps to customize labels and
layout
\item[enumitem] gives control over labels and lengths
in all kind of lists
\item[mdwlist] is useful to customize description lists, it
even allows multi-line labels. It features compact lists and
the capability to suspend and resume.
\item[desclist] offers more flexibility in definition list
\item[multenum] produces vertical enumeration in multiple
columns
\end{description}

\subsection{Layouts}

Show various lengths

\listdiagram

\section{Pages}

By default, LaTeX sets the outer margin twice as wide as the inner margin so that an opened book has three margins (left, middle, right) of equal size. The bottom margin contains the page number.

\begin{spacing}{2.4}
\noindent I want really wide spacing \\
I want really wide spacing \\
I want really wide spacing \\
\end{spacing}

\section{multilingual Support}

Here, suppose we want some good German text support.

Here I use the \code{babel} package for basic multilingual support. There is a good \href{https://en.wikibooks.org/wiki/LaTeX/Internationalization}{article}
on Wikibooks.

\begin{quotation}
\begin{otherlanguage*}{german}
  1237 und 1244 wurden die Nachbarstädte Alt-Kölln und Alt-Berlin im heutigen Ortsteil Mitte erstmals urkundlich erwähnt. Die Doppelstadt wurde als Handelsplatz gegründet und stieg im Mittelalter zu einem bedeutenden Wirtschaftszentrum auf. In seiner fast 800-jährigen Geschichte war Berlin Hauptstadt der Mark Brandenburg, Preußens und Deutschlands. Im Laufe des 18. und 19. Jahrhunderts festigte Berlin sich als internationale Großstadt mit viel Zuzug bis hin zur weltweit viertgrößten Stadt, mit der Berliner Klassik als Kulturstandort, als Zentrum der europäischen Aufklärung, sowie als bedeutender Industrie- und Wissenschaftsstandort. Nach dem Ende des Zweiten Weltkriegs unterlag die Stadt 1945 dem Viermächtestatus; Ost-Berlin hatte ab 1949 die Funktion als Hauptstadt der sozialistischen Autokratie der Deutschen Demokratischen Republik, während West-Berlin sich eng an die freiheitlich-demokratische Bundesrepublik anschloss. Mit dem Fall der Berliner Mauer 1989 und der deutschen Wiedervereinigung im Jahr 1990 wuchsen die beiden Stadthälften wieder zusammen und Berlin erhielt seine Rolle als gesamtdeutsche Hauptstadt zurück. Seit 1999 ist die Stadt Sitz der Bundesregierung, des Bundespräsidenten, des Deutschen Bundestages, des Bundesrates sowie der meisten Bundesministerien, zahlreicher Bundesbehörden und Botschaften.
\end{otherlanguage*}
\end{quotation}

I'd like some Chinese.


\begin{quotation}
\foreignlanguage{chinese}{中文测试}
\end{quotation}

Fonts packages are available for \TeX\ in additional to font files (actually the inverse?).

\pangram[\rmfamily]
\pangram[\sffamily]
\pangram[\ttfamily]
\pangram[\itshape]
\pangram[\slshape]

\end{document}
